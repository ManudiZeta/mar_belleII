%----------------------------------------------------------------------------------------
%	PACKAGES AND THEMES
%----------------------------------------------------------------------------------------
\documentclass[aspectratio=169,xcolor=dvipsnames, t]{beamer}
\usepackage{fontspec} % Allows using custom font. MUST be before loading the theme!
\usetheme{SimplePlusAIC}
\usepackage{hyperref}
\usepackage{mathtools}
\usepackage{graphicx} % Allows including images
\usepackage{booktabs} % Allows the use of \toprule, \midrule and  \bottomrule in tables
\usepackage{svg} %allows using svg figures
\usepackage{tikz}
\usepackage{makecell}
% ADD YOUR PACKAGES BELOW
\usepackage{wrapfig}
\usepackage[export]{adjustbox}

\newcommand{\backupbegin}{
   \newcounter{finalframe}
   \setcounter{finalframe}{\value{framenumber}}
}
\newcommand{\backupend}{
   \setcounter{framenumber}{\value{finalframe}}
}
%----------------------------------------------------------------------------------------
%	TITLE PAGE CONFIGURATION
%----------------------------------------------------------------------------------------

\title[short title]{$\bar{n}$ and recoil in PHSP $\gamma$ vs ISR $\gamma$} % The short title appears at the bottom of every slide, the full title is only on the title page
\subtitle{(October 21 2025 update)}

\author{Emanuele Zanusso}
\institute[Dipartimento di Fisica di Torino]{Dipartimento di Fisica
\newline
Università degli studi di Torino
}
% Your institution as it will appear on the bottom of every slide, maybe shorthand to save space


\date{October 21 2025} % Date, can be changed to a custom date
%----------------------------------------------------------------------------------------
%	PRESENTATION SLIDES
%----------------------------------------------------------------------------------------

\begin{document}

\maketitlepage

\begin{frame}[t]{Summary}
    % Throughout your presentation, if you choose to use \section{} and \subsection{} commands, these will automatically be printed on this slide as an overview of your presentation
    \tableofcontents
\end{frame}

%------------------------------------------------
% Section divider frame
\makesection{The channels studied in PHSP vs ISR}

%------------------------------------------------
% Theoretical aspects
\begin{frame}{The .dec file}

    \begin{itemize}
    
        \item The PHSP decay channel is:
        \begin{center}
        $\Upsilon(4S)  \rightarrow p \pi^- \bar{n} \gamma$ \ \ \  (evt\_gen)
	\end{center}
	
	\item The ISR decay channel is:
        \begin{center}
       $ e^+ e^- (\gamma_{ISR}) \rightarrow vpho \rightarrow p \pi^- \bar{n} $ \ \ \ (phokhara + evt\_gen)
       \end{center}
	
       \item 50k events are generated in both cases 
       \item \textbf{ISR\_events = True} in \textbf{add\_phokhara\_evtgen\_combination()} $\rightarrow$ each event contains at least one  ISR $\gamma$ 
      
        
    \end{itemize}
\end{frame}

\begin{frame}{The steering file}
    \begin{itemize}
        \item Recoil from three body system p $\pi^-$ and $\gamma$ 
        \item Cuts applied:
        \begin{enumerate}
        \item mRecoil > 0 GeV
        \item The best $\bar{n}$ candidate is selected using rankByLowest on  $\alpha$, the 3D angle between the recoil vector and the $\bar{n}$ candidate
        \item $\alpha$ < 0.35 rad ($\sim 20$ deg)
        \end{enumerate}
        
        \item MC selections applied:
        \begin{enumerate}
         \item Reconstructed particle must originate from $\Upsilon(4S)$ (PHSP) or from vpho (ISR)
         \item mcPDG selection for p (2212), $\gamma$ (22), $\pi^-$ (-211)
        \end{enumerate} 
         
	\item We'd like to discuss about variables as mRecoil, $\theta$ and p
	
        
    \end{itemize}
\end{frame}

\begin{frame}{What happens in PHSP case}
\begin{itemize}
\item PHSP case with reconstructed p, $\pi^-$, $\gamma$ and $\bar{n}$:
\begin{figure}[p]
        \includegraphics[scale=0.34, angle=0]{images/topoana_PHSP.png}
\end{figure}

\end{itemize}
\end{frame}

\begin{frame}{What happens in ISR case}
\begin{itemize}
\item ISR case with reconstructed p, $\pi^-$, $\gamma$ and $\bar{n}$:
\begin{figure}[p]

        \includegraphics[scale=0.34, angle=0]{images/topoana_ISR.png}
\end{figure}

\end{itemize}
\end{frame}


%------------------------------------------------

\makesection{Some $\bar{n}$ and $\gamma$ distributions from MC}

%------------------------------------------------

\begin{frame}{MC ISR $\gamma$ list from generator}
\begin{itemize}
\item $\gamma$ p vs $\theta$ distribution from MC:
\begin{figure}[p]
        \includegraphics[scale=0.44, angle=0]{images/theta_vs_p_ISR_gamma_MC.pdf}
\end{figure}

    \end{itemize}
\end{frame}

\begin{frame}{$\theta$ distribution for $\gamma$}
\begin{itemize}
\item $\theta_{PHSP}$ vs $\theta_{ISR}$ from MC:
\begin{figure}[p]
        \includegraphics[scale=0.44, angle=0]{images/gamma_boost.pdf}
\end{figure}


    \end{itemize}
\end{frame}
%------------------------------------------------

\begin{frame}{$\theta$ distribution for $\bar{n}$}
\begin{itemize}
\item $\theta_{PHSP}$ vs $\theta_{ISR}$ from MC:
\begin{figure}[p]
        \includegraphics[scale=0.44, angle=0]{images/nbar_boost.pdf}
\end{figure}


    \end{itemize}
\end{frame}

%------------------------------------------------

\makesection{Some $\bar{n}$ and $\gamma$ distributions from REC}

\begin{frame}{Reconstructed mRecoil}
 \begin{itemize}
\item mRecoil distribution close to $\bar{n}$ mass in both cases:

\begin{figure}[p]
        \includegraphics[scale=0.24, angle=0]{images/mRecoil_PHSP_35.pdf}
         \includegraphics[scale=0.24, angle=0]{images/mRecoil_ISR_35.pdf}
\end{figure}
\item PHSP $\rightarrow$ a distinct peak around $\bar{n}$ mass
\item ISR $\rightarrow$ braoder peak around $\bar{n}$ mass 

         \end{itemize}

\end{frame}

\begin{frame}{$\theta$ in recoil vs rec $\bar{n}$}
\begin{itemize}
\item Good correlation between them:
\begin{figure}[p]
        \includegraphics[scale=0.30, angle=0]{images/theta_corr_rec_PHSP_35.pdf}
        \includegraphics[scale=0.30, angle=0]{images/theta_corr_rec_ISR_35.pdf}
\end{figure}


    \end{itemize}
\end{frame}

%------------------------------------------------

\begin{frame}{Momentum p with reconstructed $\bar{n}$}
\begin{itemize}
\item As previously discussed, there is no correlation in momentum between the recoil and  the reconstructed $\bar{n}$:
\begin{figure}[p]
        \includegraphics[scale=0.24, angle=0]{images/p_corr_PHSP_rec.pdf}
        \includegraphics[scale=0.24, angle=0]{images/p_corr_ISR_rec.pdf}
\end{figure}

\item We could see clusterE variable (next slide)

\end{itemize}
\end{frame}
%------------------------------------------------
\begin{frame}{nbar\_clusterE variable}
\begin{itemize}
\item There is no linear correlation, but there appears a slightly better dependency between the two variables:
\begin{figure}[p]
        \includegraphics[scale=0.30, angle=0]{images/clusterE_corr_PHSP.pdf}
        \includegraphics[scale=0.30, angle=0]{images/clusterE_corr_ISR.pdf}
\end{figure}
    \end{itemize}
\end{frame}

%------------------------------------------------

\makesection{What comes}

\begin{frame}{What comes}
 \begin{itemize}
 
\item Introduce new channels as:
\begin{enumerate}
\item Channel $J/psi \rightarrow p \bar{n} \pi^- $ (work in progress)
\item Channel $\Lambda (\rightarrow \ p \pi^- ) \bar{\Lambda} (\rightarrow \bar{n} \pi^0)$ (work in progress)
\end{enumerate}
    \end{itemize}
\end{frame}


\finalpagetext{Thank you for your attention}
%----------------------------------------------------------------------------------------
\makefinalpage
%----------------------------------------------------------------------------------------
\backupbegin

\backupend
\end{document}