%----------------------------------------------------------------------------------------
%	PACKAGES AND THEMES
%----------------------------------------------------------------------------------------
\documentclass[aspectratio=169,xcolor=dvipsnames, t]{beamer}
\usepackage{fontspec} % Allows using custom font. MUST be before loading the theme!
\usetheme{SimplePlusAIC}
\usepackage{hyperref}
\usepackage{mathtools}
\usepackage{graphicx} % Allows including images
\usepackage{booktabs} % Allows the use of \toprule, \midrule and  \bottomrule in tables
\usepackage{svg} %allows using svg figures
\usepackage{tikz}
\usepackage{makecell}
% ADD YOUR PACKAGES BELOW
\usepackage{wrapfig}
\usepackage[export]{adjustbox}

\newcommand{\backupbegin}{
   \newcounter{finalframe}
   \setcounter{finalframe}{\value{framenumber}}
}
\newcommand{\backupend}{
   \setcounter{framenumber}{\value{finalframe}}
}
%----------------------------------------------------------------------------------------
%	TITLE PAGE CONFIGURATION
%----------------------------------------------------------------------------------------

\title[short title]{$\bar{n}$ and recoil (J/psi channel)} % The short title appears at the bottom of every slide, the full title is only on the title page
\subtitle{(November 11 2025 update)}

\author{Emanuele Zanusso}
\institute[Dipartimento di Fisica di Torino]{Dipartimento di Fisica
\newline
Università degli studi di Torino
}
% Your institution as it will appear on the bottom of every slide, maybe shorthand to save space


\date{November 11 2025} % Date, can be changed to a custom date
%----------------------------------------------------------------------------------------
%	PRESENTATION SLIDES
%----------------------------------------------------------------------------------------

\begin{document}

\maketitlepage

\begin{frame}[t]{Summary}
    % Throughout your presentation, if you choose to use \section{} and \subsection{} commands, these will automatically be printed on this slide as an overview of your presentation
    \tableofcontents
\end{frame}

%------------------------------------------------
% Section divider frame
\makesection{The channels}

%------------------------------------------------
% Theoretical aspects
\begin{frame}{The .dec file}

    \begin{itemize}
    
        \item The first decay channel is:
        \begin{center}
       $ e^+ e^- (\gamma_{ISR}) \rightarrow p \pi^- \bar{n} $ \ \ \ (phokhara + evt\_gen)
       \end{center}
       
          \item The second decay channel (the new one) is:
        \begin{center}
       $ e^+ e^- (\gamma_{ISR}) \rightarrow J/\psi \rightarrow p \pi^- \bar{n} $ \ \ \ (evt\_gen VECTORISR)
       \end{center}
	\begin{figure}[p]
        \includegraphics[scale=0.24, angle=0]{images/Jpsi_decfile.png}
	\end{figure}
       \item 50k events were generated in both cases 
      
        
    \end{itemize}
\end{frame}
%------------------------------------------------
\begin{frame}{The steering file}
    \begin{itemize}
        \item Recoil from three body system p $\pi^-$ and $\gamma$ 
        \item Applied cuts:
        \begin{enumerate}
        \item mRecoil > 0 GeV \&\& $\alpha$ < 0.35 rad ($\sim 20$ deg) \&\& nbar\_isFromECL == 1, where $\alpha$ is the angle between the recoil vector and the closest $\bar{n}$ candidate (rankByLowest)
        \end{enumerate}
        
        \item Applied MC selections:
        \begin{enumerate}
         \item Reconstructed particle must originate from vpho (1) or from $J/\psi$ (2)
         \item mcPDG selection for p (2212), $\gamma$ (22), $\pi^-$ (-211)
        \end{enumerate} 
	     
    \end{itemize}
\end{frame}

%------------------------------------------------
\begin{frame}{The steering file}
    \begin{itemize}
          
	\item 1C kinematic fit with for the recoil mass: 
\begin{figure}[p]
        \includegraphics[scale=0.34, angle=0]{images/kin_fit.png}
\end{figure}

	\item Where vpho:list\_rec is a dummy particle which acts as the recoil
	     
    \end{itemize}
\end{frame}

%------------------------------------------------

\makesection{Channel (2): MC distributions}

%------------------------------------------------

\begin{frame}{What happens in $J/\psi$ channel}
\begin{itemize}
\item Three ISR $\gamma$ each event:
\begin{figure}[p]
        \includegraphics[scale=0.34, angle=0]{images/topoana_Jpsi.png}
\end{figure}

\end{itemize}
\end{frame}

%------------------------------------------------

\begin{frame}{$J/\psi$ channel (VECTORISR model)}
\begin{itemize}
\item $\gamma_{ISR}$ momentum:
\begin{figure}[p]
        \includegraphics[scale=0.44, angle=0]{images/VECISR_p_MC.pdf}
\end{figure}

\end{itemize}
\end{frame}

%------------------------------------------------
\begin{frame}{$J/\psi$ channel(VECTORISR model)}
\begin{itemize}
\item $\gamma_{ISR}$ $\theta$:
\begin{figure}[p]
        \includegraphics[scale=0.44, angle=0]{images/VECISR_theta_MC.pdf}
\end{figure}

\end{itemize}
\end{frame}

%------------------------------------------------

\makesection{Channels (1) \& (2): reconstructed distributions (kinematic fit)}

\begin{frame}{Reconstructed mRecoil in channel (1)}
\begin{itemize}

\item  Recoil of p $\pi^-$ and $\gamma$ before and after the kinematic fit:
\begin{figure}[p]
 	\includegraphics[scale=0.34, angle=0]{images/mRecoil_std.pdf}
	\includegraphics[scale=0.34, angle=0]{images/mRecoil_std_kin.pdf}
\end{figure}

\end{itemize}
\end{frame}
%------------------------------------------------

\begin{frame}{Reconstructed mRecoil in channel (2)}
\begin{itemize}

\item  p $\pi^-$ and $\gamma$ recoil before and after the kinematic fit:
\begin{figure}[p]
 	\includegraphics[scale=0.34, angle=0]{images/mRecoil_Jpsi.pdf}
	\includegraphics[scale=0.34, angle=0]{images/mRecoil_Jpsi_kin.pdf}
\end{figure}

\end{itemize}
\end{frame}
%------------------------------------------------

\begin{frame}{Reconstructed mRecoil in channel (2)}
\begin{itemize}

\item $\gamma_{ISR}$ recoil is above $J/\psi$ mass in channel (2):
\begin{figure}[p]
 	\includegraphics[scale=0.34, angle=0]{images/mRecoil_gamma_Jpsi.pdf}
	\includegraphics[scale=0.34, angle=0]{images/mRecoil_gamma_Jpsi_kin.pdf}
\end{figure}

\end{itemize}
\end{frame}
%------------------------------------------------

\begin{frame}{Reconstructed $\theta$ correlation in channel (2)}
\begin{itemize}

\item Good $\theta$ correlation also observed in channel (2):
\begin{figure}[p]
 	\includegraphics[scale=0.34, angle=0]{images/theta_corr_Jpsi.pdf}
	\includegraphics[scale=0.34, angle=0]{images/theta_corr_Jpsi_kin.pdf}
\end{figure}

\end{itemize}
\end{frame}

%------------------------------------------------

\begin{frame}{Reconstructed $\Delta \theta$  in channel (2)}
\begin{itemize}

\item Good $\theta$ correlation also observed in channel (2):
\begin{figure}[p]
 	\includegraphics[scale=0.34, angle=0]{images/theta_diff_Jpsi.pdf}
	\includegraphics[scale=0.34, angle=0]{images/theta_diff_Jpsi_kin.pdf}
\end{figure}

\end{itemize}
\end{frame}

%------------------------------------------------

\makesection{Next steps}

\begin{frame}{What comes}
 \begin{itemize}
 \item Compare reconstructed $\bar{n}$ from anti-n0:list and gamma:list, studying cluster variables as $clusterLAT, clusterNHits, clusterE$ etc... (work in progress)
 
\item Introduce new channel: $\Lambda (\rightarrow \ p \pi^- ) \bar{\Lambda} (\rightarrow \bar{n} \pi^0)$ (work in progress)

\item Try to use real data (soon)

    \end{itemize}
\end{frame}





\finalpagetext{Thank you for your attention}
%----------------------------------------------------------------------------------------
\makefinalpage
%----------------------------------------------------------------------------------------
\backupbegin

\backupend
\end{document}