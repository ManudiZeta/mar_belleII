%----------------------------------------------------------------------------------------
%	PACKAGES AND THEMES
%----------------------------------------------------------------------------------------
\documentclass[aspectratio=169,xcolor=dvipsnames, t]{beamer}
\usepackage{fontspec} % Allows using custom font. MUST be before loading the theme!
\usetheme{SimplePlusAIC}
\usepackage{hyperref}
\usepackage{mathtools}
\usepackage{graphicx} % Allows including images
\usepackage{booktabs} % Allows the use of \toprule, \midrule and  \bottomrule in tables
\usepackage{svg} %allows using svg figures
\usepackage{tikz}
\usepackage{makecell}
% ADD YOUR PACKAGES BELOW
\usepackage{wrapfig}
\usepackage[export]{adjustbox}

\newcommand{\backupbegin}{
   \newcounter{finalframe}
   \setcounter{finalframe}{\value{framenumber}}
}
\newcommand{\backupend}{
   \setcounter{framenumber}{\value{finalframe}}
}
%----------------------------------------------------------------------------------------
%	TITLE PAGE CONFIGURATION
%----------------------------------------------------------------------------------------

\title[short title]{$\bar{n}$ vs recoil variables} % The short title appears at the bottom of every slide, the full title is only on the title page
\subtitle{(September 09 2025 update)}

\author{Emanuele Zanusso}
\institute[Dipartimento di Fisica di Torino]{Dipartimento di Fisica
\newline
Università degli studi di Torino
}
% Your institution as it will appear on the bottom of every slide, maybe shorthand to save space


\date{September 09 2025} % Date, can be changed to a custom date
%----------------------------------------------------------------------------------------
%	PRESENTATION SLIDES
%----------------------------------------------------------------------------------------

\begin{document}

\maketitlepage

\begin{frame}[t]{Summary}
    % Throughout your presentation, if you choose to use \section{} and \subsection{} commands, these will automatically be printed on this slide as an overview of your presentation
    \tableofcontents
\end{frame}

%------------------------------------------------
% Section divider frame
\makesection{The channel studied}

%------------------------------------------------
% Theoretical aspects
\begin{frame}{The .dec file}
    \begin{itemize}
        \item The decay channel is:
        \begin{center}
        $\Upsilon(4S)  \rightarrow p \pi^- \bar{n} \gamma$ 
	\end{center}
	
	\item It is studied as: 
	  \begin{center}
        $\Upsilon(4S)  \rightarrow v_{pho} \ \bar{n}$  \ \ \ ($v_{pho}  \rightarrow p \pi^- \gamma$) 
	\end{center}
	
       \item PHSP model used
       \item 20000 events generated
       \item Recoil from three body system $p \pi^- \gamma$
        
    \end{itemize}
\end{frame}

\begin{frame}{The steering file}
    \begin{itemize}
        \item p $\pi^- \gamma$ are reconstructed: fillParticleList used
        \item Cuts applied:
        
        \begin{enumerate}
         \item Reconstructed particle from $\Upsilon(4S)$, ($particle\_genMotherPDG == 300553$)
        	\item mRecoil > 0 GeV and mRecoil < 2 GeV
	\item $p\_mcPDG == 2212$ and $pi\_mcPDG == -211$ and $gamma\_mcPDG == 22$
        \end{enumerate} 
         	
	\item Additional cut using \textbf{rankByLowest} over the 3D angle $\alpha$ (the angle between the $\bar{n}$ candidates and the recoil vector)
	
        
    \end{itemize}
\end{frame}

%------------------------------------------------

\makesection{rankByLowest}

\begin{frame}{The $\alpha$ angle}
\begin{itemize}
\item It is the 3D angle among the $\bar{n}$ candidates and the recoil vector
\item The obtained formula is: 

\begin{center}
$\alpha = arcos[sin(\theta_1)sin(\theta_2)cos(\phi_1 - \phi_2) + cos(\theta_1)cos(\theta_2) ]$
\end{center}

(The variable \textbf{angleBetweenDaughterAndRecoil} does the same)

\item The alpha distribution is: 
\begin{figure}[p]
        \includegraphics[scale=0.30, angle=0]{images/alpha_uncorr.pdf}
\end{figure}

    \end{itemize}
\end{frame}
%------------------------------------------------

\begin{frame}{The $\alpha$ angle}
\begin{itemize}
\item Only one $\bar{n}$ candidate for each recoil vector is selected, namely the one closest to it
\item In order to achieve it, rankByLowest is adopted, where only the best ranked $\bar{n}$ candidates are saved
\item The related alpha distribution is: 
\begin{figure}[p]
        \includegraphics[scale=0.30, angle=0]{images/alpha.pdf}
\end{figure}


    \end{itemize}
\end{frame}
%------------------------------------------------
\begin{frame}{Some new relevant distributions}
\begin{itemize}
\item The updated recoil distributions:
\begin{figure}[p]
        \includegraphics[scale=0.24, angle=0]{images/vpho_mRecoil.pdf}
         \includegraphics[scale=0.24, angle=0]{images/pRecoil.pdf}
\end{figure}


    \end{itemize}
\end{frame}
%------------------------------------------------

\begin{frame}{Some new relevant distributions}
\begin{itemize}
\item The updated geometric correlations:
\begin{figure}[p]
        \includegraphics[scale=0.24, angle=0]{images/theta_corr_REC.pdf}
         \includegraphics[scale=0.24, angle=0]{images/phi_corr_REC.pdf}
\end{figure}


    \end{itemize}
\end{frame}
%------------------------------------------------

\makesection{What comes}

\begin{frame}{What comes}
 \begin{itemize}
 
\item Introduce $\gamma_{ISR}$ and compared it to $\gamma_{PHSP}$, studying the momentum distributions (\textbf{very soon})
\item Introduce new channels as:
\begin{enumerate}
\item  $\Upsilon(4S)  \rightarrow J/\Psi  \gamma_{ISR}$ (with $J/\Psi \rightarrow   p \pi^- \bar{n}$) (\textbf{started})
\item $\Upsilon(4S) \rightarrow \Lambda \bar{\Lambda} \gamma_{ISR}$ (with $\Lambda \rightarrow p \pi^-$ and $\bar{\Lambda} \rightarrow \bar{n} \pi^0$) (\textbf{soon})
\end{enumerate}
    \end{itemize}
\end{frame}


\finalpagetext{Thank you for your attention}
%----------------------------------------------------------------------------------------
\makefinalpage
%----------------------------------------------------------------------------------------
\backupbegin

\backupend
\end{document}